\documentclass{beamer}
\usetheme{Madrid}
\begin{document}
	\begin{frame}
		\frametitle{Server side Javascript}
		\begin{enumerate}
			\item Introduction to Node.js
			\begin{enumerate}
				\item Command line javascript vs	% Talk about Node as a
					Server side javascript			% javascript engine and modules
			\end{enumerate}

			\item Dependancies						% Talk about modules
													% definitions and package.json
			\begin{enumerate}
				\item Modules						% Modules in node
				\item package.json					% Talk about Node as a
													% javascript engine and modules
			\end{enumerate}

			\item Web modules: http \& Express		% Differences between the two
			\begin{enumerate}
				\item http							% Use example
				\item Express						% Use example
			\end{enumerate}

			\item Custom modules in Node.js			% how to write a module
			\begin{enumerate}
				\item Basic structure and Exports	% Use example
				\item Private methods and attributes% Use example
			\end{enumerate}
		\end{enumerate}
	\end{frame}

	\begin{frame}
		\frametitle{Server side Javascript}
		\framesubtitle{Introduction to Node.js}
		\begin{enumerate}
			\item Platform built on Chrome's Javascript engine (V8)
			\item Event-driven and Asynchronous coding
			\item Uses native Javascript
			\item Uses modules to build applications
		\end{enumerate}
	\end{frame}

	\begin{frame}[fragile]
		\frametitle{Server side Javascript}
		\framesubtitle{Introduction to Node.js - Command line javascript vs Server side javascript}

		PHP Command line:
		\begin{verbatim}
		> php -a
		Interactive mode enabled

		php > print("hello world");
		hello world
		\end{verbatim}

		Node Command line:
		\begin{verbatim}
		> node
		> console.log("Hello world");
		Hello world
		undefined
		\end{verbatim}
	\end{frame}


	\begin{frame}
		\frametitle{Server side Javascript}
		\framesubtitle{Dependancies}
		\begin{enumerate}
			\item package.json \\
				Description of the project.
			\item Modules \\
				To structure the application's code and use third party modules.
		\end{enumerate}
	\end{frame}

	\begin{frame}
		\frametitle{Server side Javascript}
		\framesubtitle{Dependancies - package.json}

		\begin{enumerate}
			\item Description file of the project (name, description, owner
				email...)
			\item Can contain dependancies with third party libraries
		\end{enumerate}
	\end{frame}

	\begin{frame}[fragile]
		\frametitle{Server side Javascript}
		\framesubtitle{Dependancies - Modules}
		Example
		\begin{verbatim}
			{
			    "name": "My Project",
			    "description": "This is an amazing project",
			    "author": "John Doe <john@doe.com>",
			    "dependencies": {
			        "express": ">= 1.2.0",
			        "socket.io": "1.x"
			    }
			}
		\end{verbatim}
	\end{frame}

	\begin{frame}
		\frametitle{Server side Javascript}
		\framesubtitle{Dependancies - Modules}

		\begin{enumerate}
			\item Third party modules (defined in package.json) are in a
				node\_modules folder.
			\item User modules can be anywhere and called with relative paths.
		\end{enumerate}
	\end{frame}

	\begin{frame}[fragile]
		\frametitle{Server side Javascript}
		\framesubtitle{Dependancies - Modules}

		Example:
		\begin{verbatim}
			{
			    "name": "websockets-springboard",
			    "version": "0.0.1",
			    "description": "spring board websocket gateway",
			    "dependencies": {
			        "bunyan": "~1.2.0",
			        "express": "~4.9.7",
			        "lodash": "^2.4.1",
			        "rabbit.js": "~0.4.1",
			        "socket.io": "~1.1.0"
			    }
			}
		\end{verbatim}
	\end{frame}

	\begin{frame}[fragile]
		\frametitle{Server side Javascript}
		\framesubtitle{Dependancies - Modules}

		Example (Suite):
		\begin{verbatim}
			server/websockets/nodejs/
			|---- node_modules
			|   |---- bunyan
			|   |---- express
			|   |---- lodash
			|   |---- rabbit.js
			|   |---- socket.io
			|---- package.json
			|---- websocket.js
			|---- wgServer.js
		\end{verbatim}
	\end{frame}

	\begin{frame}[fragile]
		\frametitle{Server side Javascript}
		\framesubtitle{Dependancies - Modules}

		Example (Suite):
		\begin{verbatim}
			// module from package.json
			var bunyan = require('bunyan');

			// builtin module
			var http = require('http');

			// Home made, application specific, module
			var wgServer = require('./wgServer');
		\end{verbatim}
	\end{frame}

	\begin{frame}
		\frametitle{Server side Javascript}
		\framesubtitle{Web modules: http \& Express}
		\begin{enumerate}
			\item Modules are needed to use node as a webserver
			\item Express uses http
		\end{enumerate}
	\end{frame}

	\begin{frame}[fragile]
		\frametitle{Server side Javascript}
		\framesubtitle{Web module: http}

		Basic HTTP calls, convenient for simple or specific applications.
		\begin{verbatim}
		var http = require('http');
		http.createServer(function (req, res) {
		    res.writeHead(200, {'Content-Type': 'text/plain'});
		    res.end('Hello World\n');
		}).listen(1337, '127.0.0.1');
		console.log('Server running at http://127.0.0.1:1337/');
		\end{verbatim}
	\end{frame}

	\begin{frame}
		\frametitle{Server side Javascript}
		\framesubtitle{Web module: Express}

		Layer above http, handle file serving, http methods, sessions... \\
		Can generate a whole web project structure with routes managements ,
		needed folders...
	\end{frame}

	\begin{frame}[fragile]
		\frametitle{Server side Javascript}
		\framesubtitle{Web module: Express}

		\begin{verbatim}
		var app = require('express')();

		app.get('/library/:file', function (req, res) {
		    // static file serving
		    res.sendFile(
		        req.params.file,
		        {
		            root: __dirname + '/library/',
		            headers: {
		                'content-type': 'application/javascript'
		            }
		        }
		    );
		});
		\end{verbatim}
	\end{frame}

	\begin{frame}
		\frametitle{Server side Javascript}
		\framesubtitle{Custom modules in Node.js}

		As seen previously, it is also possible (thanksfully!) to create our own
		modules.
	\end{frame}

	\begin{frame}[fragile]
		\frametitle{Server side Javascript}
		\framesubtitle{Custom modules in Node.js - Basic structure and Exports}

		keystone of a module: modules.exports, this must contain the public
		elements of a module.
		\begin{verbatim}
		// module definition, in [project root]/path/to/myModule.js
		var myModule = {
		    answer: 42,
		    sum: function(a, b) {
		        return a + b;
		    }
		};

		module.exports = myModule;
		\end{verbatim}
	\end{frame}

	\begin{frame}[fragile]
		\frametitle{Server side Javascript}
		\framesubtitle{Custom modules in Node.js - Basic structure and Exports}

		keystone of a module: modules.exports, this must contain the public
		elements of a module.
		\begin{verbatim}
		// usage in my application, named for example app.js
		var myNewModule = require('./path/to/myModule');

		console.log(myNewModule.answer, myNewModule.sum(1, 2));

		// run it
		> node app.js
		42 3
		\end{verbatim}
	\end{frame}

	\begin{frame}
		\frametitle{Server side Javascript}
		\framesubtitle{Custom modules in Node.js - Private methods and attributes}

		A private method/variable just has to not be included in the exports.\\
		The scope of the file is nonetheless not global, so top level variables
		will stay in the module.
	\end{frame}

	\begin{frame}[fragile]
		\frametitle{Server side Javascript}
		\framesubtitle{Custom modules in Node.js - Private methods and attributes}

		\begin{verbatim}
		// module definition, in [project root]/path/to/myModule.js
		function multiply(a, b) {
		    return a * b;
		}

		var myModule = {
		    sum: function(a, b) {
		        return a + b;
		    }
		};

		module.exports = myModule;
		\end{verbatim}
	\end{frame}

	\begin{frame}[fragile]
		\frametitle{Server side Javascript}
		\framesubtitle{Custom modules in Node.js - Private methods and attributes}

		\begin{verbatim}
		// usage in my application, named for example app.js
		var myNewModule = require('./path/to/myModule');

		console.log(myNewModule.multiply(1, 2));

		// run it
		> node app.js

		/tmp/app.js:3
		console.log(myNewModule.answer, myNewModule.multiply(1, 2));
		                                            ^
		TypeError: Object #<Object> has no method 'multiply'
		\end{verbatim}
	\end{frame}
\end{document}
