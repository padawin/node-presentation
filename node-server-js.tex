\documentclass{beamer}
\usetheme{Madrid}
\begin{document}
	\begin{frame}
		\frametitle{Server side Javascript}
		\begin{enumerate}
			\item Introduction to Node.js
			\begin{enumerate}
				\item Command line javascript vs	% Talk about Node as a
					Server side javascript			% javascript engine and modules
			\end{enumerate}

			\item Dependancies						% Talk about modules
													% definitions and package.json
			\begin{enumerate}
				\item Modules						% Modules in node
				\item package.json					% Talk about Node as a
													% javascript engine and modules
			\end{enumerate}

			\item Web modules: http \& Express		% Differences between the two
			\begin{enumerate}
				\item http							% Use example
				\item Express						% Use example
			\end{enumerate}

			\item Custom modules in Node.js			% how to write a module
			\begin{enumerate}
				\item Basic structure and Exports	% Use example
				\item Private methods and attributes% Use example
			\end{enumerate}
		\end{enumerate}
	\end{frame}

	\begin{frame}
		\frametitle{Server side Javascript}
		\framesubtitle{Introduction to Node.js}
		\begin{enumerate}
			\item Platform built on Chrome's Javascript engine (V8)
			\item Event-driven and Asynchronous coding
			\item Uses native Javascript
			\item Uses modules to build applications
		\end{enumerate}
	\end{frame}

	\begin{frame}[fragile]
		\frametitle{Server side Javascript}
		\framesubtitle{Introduction to Node.js - Command line javascript vs Server side javascript}

		PHP Command line:
		\begin{verbatim}
		> php -a
		Interactive mode enabled

		php > print("hello world");
		hello world
		\end{verbatim}

		Node Command line:
		\begin{verbatim}
		> node
		> console.log("Hello world");
		Hello world
		undefined
		\end{verbatim}
	\end{frame}


	\begin{frame}
		\frametitle{Server side Javascript}
		\framesubtitle{Dependancies}
		\begin{enumerate}
			\item package.json \\
				Description of the project.
			\item Modules \\
				To structure the application's code and use third party modules.
		\end{enumerate}
	\end{frame}

	\begin{frame}
		\frametitle{Server side Javascript}
		\framesubtitle{Dependancies - Modules}

		\begin{enumerate}
			\item Third party modules are in a node\_modules folder.
			\item User modules can be anywhere and called with relative paths.
		\end{enumerate}
	\end{frame}

	\begin{frame}[fragile]
		\frametitle{Server side Javascript}
		\framesubtitle{Dependancies - Modules}
		Example
		\begin{verbatim}
			{
				"name": "My Project",
				"description": "This is an amazing project",
				"author": "John Doe <john@doe.com>",
				"dependencies": {
					"express": ">= 1.2.0",
					"socket.io": "1.x"
				}
			}
		\end{verbatim}
	\end{frame}

	\begin{frame}
		\frametitle{Server side Javascript}
		\framesubtitle{Dependancies - package.json}
	\end{frame}
	\begin{frame}
		\frametitle{Server side Javascript}
		\framesubtitle{Web modules: http \& Express}
	\end{frame}
	\begin{frame}
		\frametitle{Server side Javascript}
		\framesubtitle{Web module: http}
	\end{frame}
	\begin{frame}
		\frametitle{Server side Javascript}
		\framesubtitle{Web module: Express}
	\end{frame}
	\begin{frame}
		\frametitle{Server side Javascript}
		\framesubtitle{Custom modules in Node.js}
	\end{frame}
	\begin{frame}
		\frametitle{Server side Javascript}
		\framesubtitle{Custom modules in Node.js - Basic structure and Exports}
	\end{frame}
	\begin{frame}
		\frametitle{Server side Javascript}
		\framesubtitle{Custom modules in Node.js - Private methods and attributes}
	\end{frame}
\end{document}
